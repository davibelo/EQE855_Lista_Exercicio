Obrigado pela correção! Vou refazer os cálculos com a função correta:

\[
S(x_1, x_2) = 7.5x_1^2 + 12x_2^2 - 3x_1^2x_2^2 + 18x_1 + 11
\]

Agora, vou calcular o gradiente, encontrar os pontos estacionários, calcular a matriz Hessiana nesses pontos, e então determinar os autovalores para avaliar a natureza dos pontos estacionários. Vou corrigir o processo e te mostrar os resultados.

Aqui estão os resultados com a função correta:

### Pontos Estacionários e Natureza:
1. **Ponto Estacionário**: \((-2, -1)\) 
   - **Autovalores da Matriz Hessiana**: \(28.92\) e \(-19.92\)
   - **Natureza**: Ponto de sela

2. **Ponto Estacionário**: \((-2, 1)\)
   - **Autovalores da Matriz Hessiana**: \(28.92\) e \(-19.92\)
   - **Natureza**: Ponto de sela

3. **Ponto Estacionário**: \((-1.2, 0)\)
   - **Autovalores da Matriz Hessiana**: \(15.00\) e \(15.36\)
   - **Natureza**: Mínimo local

4. **Ponto Estacionário**: \((2, -2)\)
   - **Autovalores da Matriz Hessiana**: \(-52.71\) e \(43.71\)
   - **Natureza**: Ponto de sela

5. **Ponto Estacionário**: \((2, 2)\)
   - **Autovalores da Matriz Hessiana**: \(-52.71\) e \(43.71\)
   - **Natureza**: Ponto de sela

Dos pontos estacionários encontrados, o ponto \((-1.2, 0)\) é um **mínimo local**, enquanto os demais são **pontos de sela**.

Se precisar de mais alguma coisa, como os gráficos solicitados no problema, estou à disposição!