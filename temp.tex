Para calcular a energia livre de Gibbs padrão (\(\Delta G^\circ\)) da reação, precisamos das energias livres de Gibbs padrão de formação (\( \Delta G_f^\circ \)) de cada um dos reagentes e produtos envolvidos na reação.

### 1. **Energia Livre de Gibbs Padrão de Formação (\(\Delta G_f^\circ\))**
A energia livre de Gibbs padrão de formação de uma substância é a mudança na energia livre de Gibbs quando 1 mol da substância é formado a partir de seus elementos nas suas formas mais estáveis, sob condições padrão (298 K e 1 atm).

Vamos considerar a reação química dada:

\[
\text{C}_2\text{H}_4 (g) + \text{H}_2 (g) \leftrightarrow \text{C}_2\text{H}_6 (g)
\]

A variação da energia livre de Gibbs padrão para essa reação (\(\Delta G^\circ\)) pode ser calculada utilizando as energias livres de Gibbs padrão de formação dos compostos:

\[
\Delta G^\circ = \sum (\Delta G_f^\circ \text{dos produtos}) - \sum (\Delta G_f^\circ \text{dos reagentes})
\]

Substituindo os valores:

\[
\Delta G^\circ = \Delta G_f^\circ (\text{C}_2\text{H}_6) - [\Delta G_f^\circ (\text{C}_2\text{H}_4) + \Delta G_f^\circ (\text{H}_2)]
\]

### 2. **Valores Típicos de \(\Delta G_f^\circ\)**
Os valores de \(\Delta G_f^\circ\) podem ser encontrados em tabelas termodinâmicas. Para o caso da reação dada, os valores típicos são:

- \(\Delta G_f^\circ (\text{C}_2\text{H}_4 \text{(g)}) \approx 68.1 \, \text{kJ/mol}\)
- \(\Delta G_f^\circ (\text{H}_2 \text{(g)}) = 0 \, \text{kJ/mol}\) (porque o \(\text{H}_2\) é um elemento na sua forma mais estável)
- \(\Delta G_f^\circ (\text{C}_2\text{H}_6 \text{(g)}) \approx -32.89 \, \text{kJ/mol}\)

### 3. **Cálculo de \(\Delta G^\circ\)**
Agora, podemos calcular \(\Delta G^\circ\) para a reação:

\[
\Delta G^\circ = \Delta G_f^\circ (\text{C}_2\text{H}_6) - [\Delta G_f^\circ (\text{C}_2\text{H}_4) + \Delta G_f^\circ (\text{H}_2)]
\]

Substituindo os valores:

\[
\Delta G^\circ = -32.89 \, \text{kJ/mol} - [68.1 \, \text{kJ/mol} + 0 \, \text{kJ/mol}]
\]

\[
\Delta G^\circ = -32.89 \, \text{kJ/mol} - 68.1 \, \text{kJ/mol}
\]

\[
\Delta G^\circ = -100.99 \, \text{kJ/mol}
\]

### 4. **Interpretação do Resultado**
O valor negativo de \(\Delta G^\circ\) indica que, sob condições padrão, a reação de hidrogenação do eteno para formar etano é espontânea.

### 5. **Aplicação no Problema**
Este \(\Delta G^\circ\) calculado pode agora ser utilizado na equação geral da energia livre de Gibbs:

\[
\Delta G = \Delta G^\circ + RT \ln \frac{n_1}{n_2 \cdot n_3}
\]

Isso nos permitirá analisar a influência das concentrações (números de mols \(n_1\), \(n_2\), e \(n_3\)) sobre a energia livre de Gibbs e, consequentemente, prever o comportamento da reação sob diferentes condições.

### 6. **Resolução Completa**
Para resolver completamente o problema de otimização, seria necessário:

- Substituir a expressão de \(\Delta G\) nas condições de minimização.
- Resolver o sistema de equações com as restrições impostas pela conservação das espécies atômicas (carbono e hidrogênio).

Se precisar que eu resolva o sistema ou desenvolva a otimização completa, posso fazer isso também.