Claro! Vamos abordar a resposta em duas partes: primeiro, explicando o conceito de energia livre de Gibbs, e em seguida, formulando um problema de otimização para o caso específico que você apresentou.

### Parte 1: O Que é Energia Livre de Gibbs?

A **energia livre de Gibbs** (\(G\)) é uma grandeza termodinâmica que mede a quantidade máxima de trabalho útil que pode ser realizado por um sistema em uma reação química ou processo físico, quando o sistema é mantido em temperatura e pressão constantes. Ela é crucial para prever a espontaneidade das reações químicas.

A variação da energia livre de Gibbs (\(\Delta G\)) durante uma reação é dada pela equação:

\[
\Delta G = \Delta H - T\Delta S
\]

Onde:
- \(\Delta H\) é a variação de entalpia (calor trocado a pressão constante).
- \(T\) é a temperatura em Kelvin.
- \(\Delta S\) é a variação de entropia (medida da desordem do sistema).

Uma reação é **espontânea** quando \(\Delta G\) é negativo, indicando que a reação pode ocorrer sem a necessidade de energia externa. Se \(\Delta G\) for positivo, a reação não é espontânea e precisaria de energia externa para ocorrer.

Para reações químicas específicas, a energia livre de Gibbs pode ser calculada usando a seguinte equação:

\[
\Delta G = \Delta G^\circ + RT \ln Q
\]

Onde:
- \(\Delta G^\circ\) é a variação da energia livre de Gibbs em condições padrão (1 atm, 298 K).
- \(R\) é a constante dos gases (8,314 J/mol·K).
- \(Q\) é o quociente de reação, que depende das concentrações ou pressões parciais dos reagentes e produtos.

### Parte 2: Formulação do Problema de Otimização

Vamos agora formular um problema de otimização onde a energia livre de Gibbs é minimizada para a reação específica que você mostrou:

\[ C_2H_4 + H_2 \leftrightarrow C_2H_6 \]

**Objetivo:**
Minimizar a energia livre de Gibbs total do sistema para determinar as quantidades de cada substância no equilíbrio.

**Função Objetivo:**

A função objetivo a ser minimizada é a energia livre de Gibbs total do sistema:

\[
G = n_1 \Delta G_f^\circ(C_2H_6) + n_2 \Delta G_f^\circ(C_2H_4) + n_3 \Delta G_f^\circ(H_2) + RT \left[ n_1 \ln \left(\frac{n_1}{n_{\text{total}}}\right) + n_2 \ln \left(\frac{n_2}{n_{\text{total}}}\right) + n_3 \ln \left(\frac{n_3}{n_{\text{total}}}\right) \right]
\]

Onde:
- \(n_1\), \(n_2\), e \(n_3\) são as quantidades molares de \(C_2H_6\), \(C_2H_4\), e \(H_2\), respectivamente.
- \(\Delta G_f^\circ\) são as energias livres de Gibbs padrão para as respectivas substâncias.
- \(n_{\text{total}} = n_1 + n_2 + n_3\) é o número total de moles no sistema.

**Restrições:**

1. **Conservação de Massa:**
   - Conservação dos átomos de carbono:
     \[
     2n_1 + 2n_2 = \text{Total de átomos de carbono disponível}
     \]
   - Conservação dos átomos de hidrogênio:
     \[
     6n_1 + 4n_2 + 2n_3 = \text{Total de átomos de hidrogênio disponível}
     \]

2. **Não-Negatividade:**
   \[
   n_1, n_2, n_3 \geq 0
   \]

**Solução do Problema de Otimização:**

A solução deste problema pode ser encontrada utilizando métodos numéricos de otimização, como Programação Não-Linear. A solução dará as quantidades molares \(n_1\), \(n_2\), e \(n_3\) que minimizam a energia livre de Gibbs do sistema, indicando o estado de equilíbrio da reação.

Esse tipo de problema é essencial para entender como as concentrações de reagentes e produtos se distribuem no equilíbrio e como diferentes condições (como temperatura e pressão) podem afetar esse equilíbrio.